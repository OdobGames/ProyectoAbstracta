\documentclass[preview]{standalone}

\usepackage[english]{babel}
\usepackage{amsmath}
\usepackage{amssymb}

\begin{document}

\begin{align*}
\text{este método sirve para estimar }$\pi$ \text{se basó en polígonos regulares inscritos y circunscritos en un círculo de diámetro unitario. Los perímetros de estos polígonos proporcionaron límites para }$\pi$\text{. Aunque hoy en día podríamos usar funciones trigonométricas para calcular estos perímetros, Arquímedes desarrolló relaciones equivalentes usando solo construcciones geométricas. Con polígonos de 96 lados, determinó que }$3^{10/71} < \pi < 3^{1/2}$.
\end{align*}

\end{document}
