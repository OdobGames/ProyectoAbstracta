\documentclass[preview]{standalone}

\usepackage[english]{babel}
\usepackage{amsmath}
\usepackage{amssymb}

\begin{document}

\begin{align*}
\text{Este método sirve para estimar } \pi \text{. Se basó en polígonos regulares} \text{inscritos y circunscritos en un círculo de diámetro unitario. Los perímetros} \text{de estos polígonos proporcionaron límites para } \pi \text{. Aunque hoy en día} \text{podríamos usar funciones trigonométricas para calcular estos perímetros,} \text{Arquímedes desarrolló relaciones equivalentes usando solo construcciones geométricas.} \text{Con polígonos de 96 lados, determinó que } 3^{10/71} < \pi < 3^{1/2} \text{.}
\end{align*}

\end{document}
