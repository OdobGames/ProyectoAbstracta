\documentclass[preview]{standalone}

\usepackage[english]{babel}
\usepackage{amsmath}
\usepackage{amssymb}

\begin{document}

\begin{align*}
\text{Ramanujan fue incomparable en su habilidad para calcular} \text{ estos valores singulares. Uno de sus más famosos es el valor} \text{ cuando } p \text{ es igual a } 210 \text{, que fue incluido en su carta original a G. H. Hardy. Es} \text{\\} k_{210} =  \left(\sqrt{2}-1\right)^2 \left(2-\sqrt{3}\right) \left(\sqrt{7}-\sqrt{16}\right)^2 \left(8-3\sqrt{7}\right) \left(\sqrt{10}-\sqrt{3}\right)^2 \left(\sqrt{15}-\sqrt{14}\right) \left(4-\sqrt{15}\right) \text{\\} \text{Este número, cuando se sustituye en la expresión logarítmica,} \text{ coincide con } \pi \text{ hasta las primeras 20 cifras decimales. En comparación, } k_2 \text{ produce un número que coincide con } \pi \text{ a través de más de un millón de dígitos.}
\end{align*}

\end{document}
